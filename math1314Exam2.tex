% Exam Template for San Jacinto Community College Math Department courses
%Copyright@Chandi Bhandari
%%%%%%%%%%%%%%%%%%%%%%%%%%%%%%%%%%%%%%%%%%%%%%%%%%%%%%%%%%%%%%%%%%%%%%%%%%%%%%%%%%%%%%%%%%%%%%

% These lines can probably stay unchanged, although you can remove the last
% two packages if you're not making pictures with tikz.
\documentclass[11pt]{exam}
\RequirePackage{amssymb, amsfonts, amsmath, latexsym, verbatim, xspace, setspace}
\RequirePackage{tikz, pgflibraryplotmarks}
\usepackage{graphicx}

\graphicspath{ {./images/} }

% By default LaTeX uses large margins.  This doesn't work well on exams; problems
% end up in the "middle" of the page, reducing the amount of space for students
% to work on them.
\usepackage[margin=1in]{geometry}


% Here's where you edit the Class, Exam, Date, etc.
\newcommand{\class}{Math 1314 College Algebra}
\newcommand{\term}{Fall 2018}
\newcommand{\examnum}{Exam II}
\newcommand{\examdate}{September, 2018}
\newcommand{\timelimit}{90 Minutes}

% For an exam, single spacing is most appropriate
\singlespacing
% \onehalfspacing
% \doublespacing

% For an exam, we generally want to turn off paragraph indentation
\parindent 0ex

\begin{document} 

% These commands set up the running header on the top of the exam pages
\pagestyle{head}
\firstpageheader{}{}{}
\runningheader{\class}{\examnum\ - Page \thepage\ of \numpages}{\examdate}
\runningheadrule

\begin{flushright}
\begin{tabular}{p{3.2in} r l}
\textbf{\class} & \textbf{\term} & \textbf{\examnum}\\
\textbf{ID-Number-} & \textbf{Name (Print):}  \makebox[1.2in]{\hrulefill}\\
%\textbf{Time: \timelimit} % & \makebox[2in]{\hrulefill}

\end{tabular}\\
\end{flushright}
\rule[1ex]{\textwidth}{.1pt}

\hfill


%%%%%%%%%%%%%%%%%%%%%%%%%%%%%%%%%%%%%%%%%%%%%%%%%%%%%%%%%%%%%%%%%%%%%%%%%%%%%%%%%%%%%
%For the further information 
%%%%%%%%%%%%%%%%%%%%%%%%%%%%%%%%%%%%%%%%%%%%%%%%%%%%%%%%%%%%%%%%%%%%%%%%%%%%%%%%%%%%%

\begin{questions}

% Basic question
\addpoints
\question[10] Find the Equation of the line:
\begin{parts}
	\part Through (-1,4) and parallel to $x+3y=5$.
	\vspace{5cm}
	\part Passing through (1,2) and (-2,4). 
\end{parts}
\vspace{5cm}
%\newpage
\addpoints
\question[5] Find the value of $f(-5),f(0),f(3)$ for the following function.

\begin{center}
	\[f(x)= \begin{cases} 
	x-2 & for \quad  x< 3 \\
	5-x & for \quad x\geq3 
	\end{cases}
	\]
\end{center}
\vspace{9cm}
\addpoints
\question[14] 
\begin{parts}

\part [6] Graph the following piecewise function:
\begin{center}
	\[f(x)= \begin{cases} 
	x^3+3 & for \quad  x\leq 0 \\
	-x^2 & for \quad x>0 
	\end{cases}
	\]
\end{center}
\vspace{9cm}
\part [7] Graph the following piecewise function:
\begin{center}
	\[f(x)= \begin{cases} 
	\frac{1}{2}x^2+2 & for \quad  x\leq 2 \\
	\frac{1}{2}x & for \quad x>2 
	\end{cases}
	\]
\end{center}
 \end{parts}
\vspace{9cm}
\addpoints
\question [16] For $f(x)=x^2-10x+21$, find the following
\vspace{2cm}
\begin{parts}
\part  Vertex. 
\vspace{1cm}
\part  Axis
\vspace{1cm}
\part  Domain.
\vspace{1cm}
\part Range
\vspace{1cm}
\part Decreasing Interval
\vspace{1cm}
\part Increasing Interval
\vspace{1cm}
\part Graph the function
\end{parts}

\vspace{9cm}
\addpoints
\question [15] For $f(x)=\sqrt{x}$, and $g(x)=x+3$

\begin{parts}
	\part  $f\circ g(x)$ 
	\vspace{4cm}
	\part  $g\circ f(x)$ 
	\vspace{4cm}
	\part  Also, find their domain.
	\vspace{4cm}
	\part  $ (f+g)x$ 
\end{parts}



\newpage
\addpoints
\question[10] For the part a) and b)

\begin{parts}
\part Using Synthetic Division find the Quotient and Remainder
\[\frac{x^5+3x^4+2x^3+2x^2+3x+1}{x+2}\]
\vspace{8cm}
\part Using Synthetic Division find remainder and decide whether $x-1$ is the factor or not
\[\frac{2x^3+9x^2-16x+12}{x-1}\]
\vspace{8cm}
\end{parts}

\vspace{8cm}
\addpoints
\question[10] Check whether the following function are even, odd or neither. Also check for the symmetry of the graph (must check on all three).
\[y=x^2-x+8\]

\newpage
\addpoints
\question[10] For $f(x)=4x+3$ 
\begin{parts}
	\part Find $f(-1)$ and 
	\vspace{3cm}
	\part  $\frac{f(x+h)-f(x)}{h}$

\end{parts}
\vspace{8cm}
\addpoints
\question[10] Sketch the following graphs by using the graphing techniques the part a) and b)

\begin{parts}
	\part Graph 
	\[f(x)=-\frac{1}{2}(x+3)^2-4\]
	\vspace{8cm}
	\part Give the domain and graph \[f(x)=\frac{1}{x-4}+5\]
	\vspace{8cm}
\end{parts}
\vspace{8cm}
\addpoints
Bonus
\question[5] Graph the following function. Also give the domain and range \[f(x)=3\sqrt[3]{x-4}+3\]
\end{questions}
\end{document}