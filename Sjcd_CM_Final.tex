% Exam Template for San Jacinto Community College Math Department courses
%Copyright@Chandi Bhandari
%%%%%%%%%%%%%%%%%%%%%%%%%%%%%%%%%%%%%%%%%%%%%%%%%%%%%%%%%%%%%%%%%%%%%%%%%%%%%%%%%%%%%%%%%%%%%%

% These lines can probably stay unchanged, although you can remove the last
% two packages if you're not making pictures with tikz.
\documentclass[11pt]{exam}
\RequirePackage{amssymb, amsfonts, amsmath, latexsym, verbatim, xspace, setspace}
\RequirePackage{tikz, pgflibraryplotmarks}
\usepackage{graphicx}
\usepackage{venndiagram}
\graphicspath{ {./images/} }

% By default LaTeX uses large margins.  This doesn't work well on exams; problems
% end up in the "middle" of the page, reducing the amount of space for students
% to work on them.
\usepackage[margin=1in]{geometry}


% Here's where you edit the Class, Exam, Date, etc.
\newcommand{\class}{Math 1332 Contemporary Math}
\newcommand{\term}{Fall 2018}
\newcommand{\examnum}{Final Exam}
\newcommand{\examdate}{Sept, 2018}
\newcommand{\timelimit}{120 Minutes}

% For an exam, single spacing is most appropriate
\singlespacing
% \onehalfspacing
% \doublespacing

% For an exam, we generally want to turn off paragraph indentation
\parindent 0ex

\begin{document} 

% These commands set up the running header on the top of the exam pages
\pagestyle{head}
\firstpageheader{}{}{}
\runningheader{\class}{\examnum\ - Page \thepage\ of \numpages}{\examdate}
\runningheadrule

\begin{flushright}
\begin{tabular}{p{3.2in} r l}
\textbf{\class} & \textbf{\term} & \textbf{\examnum}\\
\textbf{G-Number-} & \textbf{Name (Print):}  \makebox[1.2in]{\hrulefill}\\
%\textbf{Time: \timelimit} % & \makebox[2in]{\hrulefill}

\end{tabular}\\
\end{flushright}
\rule[1ex]{\textwidth}{.1pt}

\hfill


%%%%%%%%%%%%%%%%%%%%%%%%%%%%%%%%%%%%%%%%%%%%%%%%%%%%%%%%%%%%%%%%%%%%%%%%%%%%%%%%%%%%%
%For the further information 
%%%%%%%%%%%%%%%%%%%%%%%%%%%%%%%%%%%%%%%%%%%%%%%%%%%%%%%%%%%%%%%%%%%%%%%%%%%%%%%%%%%%%

\begin{questions}

% Basic question
\addpoints
\question[6] Identify the premises, conclusion, and the type of Fallacies 

\begin{parts}
	\part I ate oysters for dinner and later that night I had a nightmare. Oyster caused my nightmare
	\vspace{5cm}
	\part Apple's iPhones outsell all other smart phones, so they must be the best smart phones on the market.  
\end{parts}

\vspace{5cm}
%\newpage
\addpoints
\question[10] For the two statements p, q find the following logical argument truth table for Negation:  not p ($\sim{p}$), Conjunction: p and q ($p\land q$), Disjunction: p or q ($p\lor q$), Conditional: if p then q ($p\implies q$), Inverse: if not p then not q ($\sim{p}\implies \sim q$), Converse: if q then p ($q\implies p$), Contrapositive: if not q then not p ($\sim{q}\implies \sim p$) .



 
\newpage
%\newpage
\addpoints
\question[12] A movie critics reviewed 36 films: 12 were documentaries and 24 were feature films. She gave favorable reviews to 18 of the 24 were feature films. She gave favorable reviews to 8 of the documentaries and unfavorable reviews to 6 of the feature films.
\begin{parts}
\part  Make a two-way table summarizing the reviews.
\vspace{5cm}
\part  Make a Venn diagram from the table in part (a).
\vspace{5cm}
\part How many documentaries received unfavorable reviews? 
\vspace{5cm}
\part How many feature films received favorable reviews?
\end{parts}

\vspace{15cm}
\addpoints
\question [8] Following is the Venn-diagram of the people at conference. Use the Venn Diagram to answer the following questions

\begin{venndiagram3sets}[labelOnlyAB={20},
	labelOnlyAC={8},
	labelOnlyBC={16},
	labelABC={11},
	labelOnlyA={4},
	labelOnlyC={6},
	labelOnlyB={9},
	labelNotABC={3},
	radius=3cm,
	overlap=2.5cm]
	\setpostvennhook
	{
		\draw[<-] (labelA) -- ++(135:3cm) node[above] {Woman};
		\draw[<-] (labelB) -- ++(45:3cm) node[above] {College degree};
		\draw[<-] (labelC) -- ++(-90:3cm) node[below] {Currently employed};
	}
\end{venndiagram3sets}




\begin{parts}
\part  How many people at the conference are employed men with a college degree?
\vspace{1cm}
\part  How many people at the conference are employed men?
\vspace{1cm}
\part   How many people at the conference are unemployed woman?
\vspace{1cm}
\part   How many women are at the conference?
\vspace{1cm}

\end{parts}



\newpage

\question[6] For the following income and expenses compute the Net Cash flow: \\
Income:\\
Part-time Job = $\$650$ per month\\
College fund from grandparents: $\$400$ per month\\
Scholarship: $\$6000$ per year. \\

Expenses: \\
Rent: $\$500$ per month.\\
Groceries: $\$60 $ per week.\\ 
Tuition and Fees: $\$3600$ twice a year.\\
Incidentals: $\$120$ per week \\


\newpage
\addpoints
\question[7] Make a frequency table and then find cumulative, relative frequency also present them in  Bar Diagram for the following data\\
A, C, D, F, H, C, C, C, A, C, H, D, D, C, A, F, H, H, H, H, A, A, D, D, A, A, C, D. 
\vspace{10cm}

\question[6] If interest rates stay at $4\%$ APR and I continue to make my monthly $\$25$ deposit into my retirement plan, how much total money I would have after 30 years. 
\vspace{10cm}

 
\vspace{10cm}
\question[6] Your goal is to create a college fund for your child. Suppose you find a fund that offers an APR of $5\%$ How much should you deposit monthly to accumulate $\$170,000$ in 15 years? 
\vspace{10cm}


\question[5] The following data represents the number of boys and girls attended at San Jacinto College from 2010 to 2015 in College Comtemporary math classes. Make a multiple bar diagram (single or double) for these data, with vertical axes representing the number of students running from 1 to 15.  \begin{table}[h]                           
	\centering
	\begin{tabular}{|c|c|c|}
		\hline
		\multicolumn{3}{c|}{\textbf{Number of boys and girls from 2010-2015}} \\ \hline
		
		Academic year    &   Number of boys   &   Number of girls      \\ \hline
		2010 &   12 &   14             \\ \hline          
		2011    &   9 &   16  \\ \hline
		2012       &  14 &   10           \\ \hline
		2013  &   13 &   8              \\ \hline
		2014       &  6  &   10   \\ \hline
		2015        &   6 &   12          \\ \hline
		
	\end{tabular}
	\caption{Students log table in Contemporary math class}
	\label{tab:msg1}                            
	
\end{table}
\vspace{10cm}



\addpoints
\question[5] Find the mean, median,and mode for following data and also find outliers if exits: 7, 3, 3, 11, 12, 3, 4, 14, 6, 4, 3, 53, 4,14, 6 

\vspace{3cm}

\question[8] Find the five point summary, range and standard deviation of the following data:12, 7, 9, 10, 7, 8

\newpage
\addpoints
\question[3] Find the probability distribution table for the sample space when tossing two coins.



\vspace{9cm}
\addpoints
\question[6]  When forming a committee of three members consisting boys and girls then find 
\begin{parts}
	\part All girs
	\vspace{1cm}
	\part Exactly two boys
	\vspace{1cm} 
	\part At least two boys
\end{parts}
\vspace{2cm}

\question[6] Find the following "AND" probability
\begin{parts}
	\part When rolling a die in one hand and tossing a coin in another hand then what is the probability of  P(Even numbers and T)
	\vspace{2cm}
	\part There are 10 tennis balls in a balls of two different colors 6 red and 4 White balls. John wants to take out two balls from that bag then what is the probability  P(White and Red)=P(W and R)
	
\end{parts}
\vspace{3cm}



\addpoints
\question[6]  There are 20 cards in a deck of card numbering from 1 through 20 that is $S=\{1, 2, 3, ...,20\}$. And Events A=$\{Even numbers\}$, B=$\{multiple of 5\}$ and $C=\{3,7, 13\}$ then find the following probability
\begin{parts}
	\part P(A or B)
	\vspace{4cm}
	\part P(A or C)
	\vspace{4cm}
\end{parts}

Bonus:
\addpoints
\question[5] The population of Pearland in 2010 was 10,000 and it  is doubling up in every 10 years then what would be population of Pearland in 2050?

\end{questions}
\end{document}