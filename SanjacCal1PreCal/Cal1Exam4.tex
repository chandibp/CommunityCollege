% Exam Template for San Jacinto Community College Math Department courses
%Copyright@Chandi Bhandari
%%%%%%%%%%%%%%%%%%%%%%%%%%%%%%%%%%%%%%%%%%%%%%%%%%%%%%%%%%%%%%%%%%%%%%%%%%%%%%%%%%%%%%%%%%%%%%

% These lines can probably stay unchanged, although you can remove the last
% two packages if you're not making pictures with tikz.
\documentclass[11pt]{exam}
\RequirePackage{amssymb, amsfonts, amsmath, latexsym, verbatim, xspace, setspace}
\RequirePackage{tikz, pgflibraryplotmarks}
\usepackage{graphicx}
\graphicspath{ {./images/} }

% By default LaTeX uses large margins.  This doesn't work well on exams; problems
% end up in the "middle" of the page, reducing the amount of space for students
% to work on them.
\usepackage[margin=1in]{geometry}


% Here's where you edit the Class, Exam, Date, etc.
\newcommand{\class}{Math 2413 Calculus-I}
\newcommand{\term}{Summer 2018}
\newcommand{\examnum}{Exam III}
\newcommand{\examdate}{July, 2018}
\newcommand{\timelimit}{90 Minutes}

% For an exam, single spacing is most appropriate
\singlespacing
% \onehalfspacing
% \doublespacing

% For an exam, we generally want to turn off paragraph indentation
\parindent 0ex

\begin{document} 

% These commands set up the running header on the top of the exam pages
\pagestyle{head}
\firstpageheader{}{}{}
\runningheader{\class}{\examnum\ - Page \thepage\ of \numpages}{\examdate}
\runningheadrule

\begin{flushright}
\begin{tabular}{p{3.2in} r l}
\textbf{\class} & \textbf{\term} & \textbf{\examnum}\\
\textbf{G-Number-} & \textbf{Name (Print):}  \makebox[1.2in]{\hrulefill}\\
%\textbf{Time: \timelimit} % & \makebox[2in]{\hrulefill}

\end{tabular}\\
\end{flushright}
\rule[1ex]{\textwidth}{.1pt}

\hfill


%%%%%%%%%%%%%%%%%%%%%%%%%%%%%%%%%%%%%%%%%%%%%%%%%%%%%%%%%%%%%%%%%%%%%%%%%%%%%%%%%%%%%
%For the further information 
%%%%%%%%%%%%%%%%%%%%%%%%%%%%%%%%%%%%%%%%%%%%%%%%%%%%%%%%%%%%%%%%%%%%%%%%%%%%%%%%%%%%%

\begin{questions}

% Basic question
\addpoints
\question[10] For the given function $f(x)=-2x^2+4x-2$, find the critical points, classify the critical points (local maximum or local minimum) also find the relative extreme values. 






\newpage
% Question with parts
%\newpage
\addpoints
\question[20] Find the Domain, Vertical asymptotes, Horizontal asymptotes, intercepts of the graph. Find all the critical points, interval of increasing and decreasing, locate fhe point at which $f(x)$ has local maximum and local minimum. Find all the point of inflection and also determine the interval on which the graph of the function is concave up and concave down for the graph Also Sketch the graph. \[f(x)=\frac{x^2-1}{2x-1}\]





\newpage
\addpoints
\question[5] Evaluate the infinite limit
\[ \lim\limits_{x\to \infty} \frac{5x^3+1}{10x^3-3x^2+7} \]






\vspace{8cm}
\addpoints
\question[10]A farmer plans to fence a rectangular pasture adjacent to a river. The pasture must contain $405,000 m^2$ in order to provide enough grass for the herd. No fencing is needed along the river. What dimension will require the least amount of fencing?



\vspace{8cm}
\newpage
\addpoints
\question[15] For the function $f(x)=4-x^2$ on the interval [-2,2] with 4 equal intervals. find the following 
\begin{parts}
	\part Lower sums $L_p(f)$
	\part Upper sums $U_p(f)$
	\part Sums at the left end points
	\part Sums at the mid-point
\end{parts}



\vspace{8cm}
\newpage
\addpoints
\question[5] Calculate the three Iteration of Newton method to approximate a zero of the function using the given initial guess. $x_1=2$. \[f(x)=x^2-5\]


\vspace{9cm}
\addpoints
\question[8] Approxite the value of $\sqrt{65}$ by using the differential. 

\newpage
\addpoints
\question[10] Find the following integration
\begin{parts}
\part $\int_{1}^{3} (9x^3-2x^2-6)dx	$
\vspace{5cm}
\part $\int_{0}^{2} (\sqrt{x}+2x+4)dx$
\vspace{4cm}
\part $\int 2sinx-4cosx+2x dx$
\end{parts}


\vspace{8cm}
\addpoints
\question[7] Find the solution of the differential equation $f'(x)=10x-12x^3, f(3)=2$


\vspace{8cm}
\addpoints
\question[10] Find the interval of concavity (concave upward and concave downward) for the function 
\[f(x)=sinx+cosx\]


\vspace{8cm}
\noaddpoints
\question[5 Bonus] Evaluate the indefinite integral \[\int \frac{x}{\sqrt{1-x^2}}dx\]
%\question[5 Bonus] Use Squeez theorem to compute the limit \[\lim\limits_{x\to \infty}\frac{cos(x)}{x}\]
\end{questions}
\end{document}