% Exam Template for San Jacinto Community College Math Department courses
%Copyright@Chandi Bhandari
%%%%%%%%%%%%%%%%%%%%%%%%%%%%%%%%%%%%%%%%%%%%%%%%%%%%%%%%%%%%%%%%%%%%%%%%%%%%%%%%%%%%%%%%%%%%%%

% These lines can probably stay unchanged, although you can remove the last
% two packages if you're not making pictures with tikz.
\documentclass[11pt]{exam}
\RequirePackage{amssymb, amsfonts, amsmath, latexsym, verbatim, xspace, setspace}
\RequirePackage{tikz, pgflibraryplotmarks}
\usepackage{graphicx}
\graphicspath{ {./images/} }

% By default LaTeX uses large margins.  This doesn't work well on exams; problems
% end up in the "middle" of the page, reducing the amount of space for students
% to work on them.
\usepackage[margin=1in]{geometry}


% Here's where you edit the Class, Exam, Date, etc.
\newcommand{\class}{Math 2413 Calculus-I}
\newcommand{\term}{Summer 2018}
\newcommand{\examnum}{Exam II}
\newcommand{\examdate}{July, 2018}
\newcommand{\timelimit}{90 Minutes}

% For an exam, single spacing is most appropriate
\singlespacing
% \onehalfspacing
% \doublespacing

% For an exam, we generally want to turn off paragraph indentation
\parindent 0ex

\begin{document} 

% These commands set up the running header on the top of the exam pages
\pagestyle{head}
\firstpageheader{}{}{}
\runningheader{\class}{\examnum\ - Page \thepage\ of \numpages}{\examdate}
\runningheadrule

\begin{flushright}
\begin{tabular}{p{3.2in} r l}
\textbf{\class} & \textbf{\term} & \textbf{\examnum}\\
\textbf{G-Number-} & \textbf{Name (Print):}  \makebox[1.2in]{\hrulefill}\\
%\textbf{Time: \timelimit} % & \makebox[2in]{\hrulefill}

\end{tabular}\\
\end{flushright}
\rule[1ex]{\textwidth}{.1pt}

\hfill


%%%%%%%%%%%%%%%%%%%%%%%%%%%%%%%%%%%%%%%%%%%%%%%%%%%%%%%%%%%%%%%%%%%%%%%%%%%%%%%%%%%%%
%For the further information 
%%%%%%%%%%%%%%%%%%%%%%%%%%%%%%%%%%%%%%%%%%%%%%%%%%%%%%%%%%%%%%%%%%%%%%%%%%%%%%%%%%%%%

\begin{questions}

% Basic question
\addpoints
\question[10] Find the Equation of Tangent and Normal line for the graph of the function $f(x)=x^4-2x^2+3$  at $x=2$. Also find the point at which the Tangent of this graph is horizontal. 






\newpage
% Question with parts
%\newpage
\addpoints
\question[12] Find the derivative of the following

\begin{parts}
\part  $f(x)=sin(2x)cos(2x)$
\vspace{2cm}
\part  $g(x)=(3x-4)(x^3+5)$
\vspace{2cm}
\part  $h(x)=\frac{3x^2-1}{2x+5}$
\vspace{5cm}
\part  Find the derivative of the function $f(x)$ at $x=3$ if it exists
\begin{center}
	\[f(x)= \begin{cases} 
	x^2 & for \quad  x\geq 3 \\
	2x+3 & for \quad x<3 
	\end{cases}
	\]
\end{center}
\vspace{3cm}
\part  $\phi(x)=\sqrt{4-3x^2}$
\vspace{3cm}
\part  $\psi(x)=\sqrt{\frac{2x}{x+1}}$
\vspace{1cm}
\end{parts}



\newpage
\addpoints
\question[15] 


\noaddpoints % If you remove this line, the grading table will show 20 points for this problem.
\begin{parts}
	\part[5] Find the derivative of the function $x^3y^3-y=x$
	\vspace{8cm}
\part[10] Find the Equation of tangent line to the graph of $x^3+y^3=6xy-1$ at $(2,3)$.
\end{parts}

\newpage

%\vspace{8cm}
\addpoints
\question[10] Water is pumped into a cylindrical tank at the rate of 240 cubic inch per second. While the height of the tank is 3 times the radius then at what rate the height is changing when the height is 5 inch. (Note: volume of cylinder is V=$\pi r^2 h$)



\vspace{8cm}
\addpoints
\question[13] A spherical balloon is inflated with gas at the rate of 800 cubic centimeters per minute. Find the rate of change pf the radius when $r=30$cm. Explain why the rate of change of the radius of the sphere is not constant even though te rate of change of volume is constant.



\vspace{8cm}
\newpage
\addpoints
\question[10] Is the Rolles' Theorem can be applied to $f(x)=(x-2)^2(x-3)$  on the interval $[2,3]$ or Not? Give reason.  If Rolle's Theorem can be applied, find all numbers c in the open interval (2,3)such that $f'(c)=0$.


\vspace{9cm}
\addpoints
\question[10]Find the Absolute extrema of the function on the closed interval [-2,1]
\begin{align*}
g(x)=\frac{6x^2}{x-2}
\end{align*}

\newpage
\addpoints
\question[10] Find the point on the interval (0, 6) at which the tangent line is parallel to the secant line of the graph $f(x)=x^2-2x+2$.


\vspace{8cm}
\addpoints
\question[10] Find all the critical points of the function $f(x)=2sin(x)-cos(2x)$ [Note: Sin(2x)-2sin(x)cos(x)].


\vspace{8cm}
\addpoints
\question[5 Bonus] Evaluate the derivative $y=sin(xy)$
\end{questions}
\end{document}