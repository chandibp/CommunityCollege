% Exam Template for San Jacinto Community College Math Department courses
%Copyright@Chandi Bhandari
%%%%%%%%%%%%%%%%%%%%%%%%%%%%%%%%%%%%%%%%%%%%%%%%%%%%%%%%%%%%%%%%%%%%%%%%%%%%%%%%%%%%%%%%%%%%%%

% These lines can probably stay unchanged, although you can remove the last
% two packages if you're not making pictures with tikz.
\documentclass[11pt]{exam}
\RequirePackage{amssymb, amsfonts, amsmath, latexsym, verbatim, xspace, setspace}
\RequirePackage{tikz, pgflibraryplotmarks}
\usepackage{graphicx}
\graphicspath{ {./images/} }

% By default LaTeX uses large margins.  This doesn't work well on exams; problems
% end up in the "middle" of the page, reducing the amount of space for students
% to work on them.
\usepackage[margin=1in]{geometry}


% Here's where you edit the Class, Exam, Date, etc.
\newcommand{\class}{Math 2413 Calculus-I}
\newcommand{\term}{Summer 2018}
\newcommand{\examnum}{Exam 1}
\newcommand{\examdate}{July, 2018}
\newcommand{\timelimit}{90 Minutes}

% For an exam, single spacing is most appropriate
\singlespacing
% \onehalfspacing
% \doublespacing

% For an exam, we generally want to turn off paragraph indentation
\parindent 0ex

\begin{document} 

% These commands set up the running header on the top of the exam pages
\pagestyle{head}
\firstpageheader{}{}{}
\runningheader{\class}{\examnum\ - Page \thepage\ of \numpages}{\examdate}
\runningheadrule

\begin{flushright}
\begin{tabular}{p{3.2in} r l}
\textbf{\class} & \textbf{\term} & \textbf{\examnum}\\
\textbf{G-Number-} & \textbf{Name (Print):}  \makebox[1.2in]{\hrulefill}\\
%\textbf{Time: \timelimit} % & \makebox[2in]{\hrulefill}

\end{tabular}\\
\end{flushright}
\rule[1ex]{\textwidth}{.1pt}

%\bf{ Choose Any 10 questions but Questions 5 and 7 are mandatory}

\hfill


%%%%%%%%%%%%%%%%%%%%%%%%%%%%%%%%%%%%%%%%%%%%%%%%%%%%%%%%%%%%%%%%%%%%%%%%%%%%%%%%%%%%%
%For the further information 
%%%%%%%%%%%%%%%%%%%%%%%%%%%%%%%%%%%%%%%%%%%%%%%%%%%%%%%%%%%%%%%%%%%%%%%%%%%%%%%%%%%%%

\begin{questions}

% Basic question
\addpoints
\question[15] Find the value of unknown $a$ such that the function $f$ is continuous at the given points


\begin{center}
\[f(x)= \begin{cases} 
3x^2 & for \quad  x\geq 1 \\
ax-4 & for \quad x<1 
\end{cases}
\mbox{at  x=1}
\]
\end{center}




\newpage
% Question with parts
%\newpage
\addpoints
\question[20] Evaluate the following limits

\begin{parts}
\part  $\lim\limits_{x\to 0}\frac{sinx}{5x}=$
\vspace{1cm}
\part  $\lim\limits_{t\to 0}\frac{sin2t}{3t}=$
\vspace{1cm}
\part  $\lim\limits_{x\to 2}\sqrt[3]{12x+3}=$
\vspace{1cm}
\part  $\lim\limits_{x\to -1}f(x)$ if it exists for the given function
\begin{center}
	\[f(x)= \begin{cases} 
	3x^2 & for \quad  x\geq -1 \\
	2x+4 & for \quad x<-1 
	\end{cases}
	\]
\end{center}
\vspace{1cm}
\part  $\lim\limits_{x\to 5}\frac{x-5}{x^2-25}$
\vspace{2cm}
\part  $\lim\limits_{x\to 4}\frac{\sqrt{x+5}-3}{x-4}$
\vspace{1cm}
\end{parts}



\newpage
\addpoints
\question[15] 


\noaddpoints % If you remove this line, the grading table will show 20 points for this problem.
\begin{parts}
	\part[5] Write down the definition of the derivative of the function $f(x)$
	\vspace{4cm}
\part[10] Find the derivative of $f(x)=x^2-5$ by using the definition of derivative.
\end{parts}

\newpage

%\vspace{8cm}
\addpoints
\question[20] For the function f(x) given in the graph, evaluate the following (if they exit)

\begin{center}
	\includegraphics{limconti1.png}	
\end{center}
\begin{parts}
	\part  $\lim\limits_{x\to -2}f(x)=$
	\vspace{1cm}
	\part  $\lim\limits_{x\to 1}f(x)=$
	\vspace{1cm}
	\part  $\lim\limits_{x\to 2}f(x)=$
	\vspace{1cm}
	\part  $\lim\limits_{x\to -1}f(x)=$ 
	\vspace{1cm}
	\part  $\lim\limits_{x\to3}f(x)=$
	\vspace{2cm}
	\part  Is $f(x)$ is continuous at x=-2 if not specify the types of discontinuity.
	\vspace{1cm}
	\part  Is $f(x)$ is continuous at x=0 if not specify the types of discontinuity.
	\vspace{1cm}
	\part  Is $f(x)$ is continuous at x=1 if not specify the types of discontinuity.
	\vspace{1cm}
	\part  Is $f(x)$ is continuous at x=2 if not specify the types of discontinuity.
	\vspace{1cm}
	\part  Is $f(x)$ is continuous at x=3 if not specify the types of discontinuity.
	
\end{parts}

\vspace{8cm}
\addpoints
\question[10] Find the point x at which $f(x)$ is not continuous, then explain which types of discontinuity it has?

\noaddpoints % If you remove this line, the grading table will show 20 points for this problem.
\begin{parts}
	\part[4] \begin{align*}
	f(x)=\frac{4}{x-6}
	\end{align*}
	\vspace{8cm}
	\part[6]  	\[f(x)= \begin{cases} 
	x^2 & for \quad  x< 1 \\
	x+1 & for \quad x\geq 1 
	\end{cases}
	\]
\end{parts}

\vspace{8cm}
\newpage
\addpoints
\question[10] Is piecewise function f(x) is continuous or not at x=0? Give reason why it is continuous or why not?
\begin{center}
	\[f(x)= \begin{cases} 
	5x^2+3x+1 & for \quad  x< 0 \\
	x+1 & for \quad x\geq 0 
	\end{cases}
	\]
\end{center}

\vspace{8cm}
\addpoints
\question[10] Explain why the function $f(x)=x^3+5x-3$ has at least one zero in the given interval [0,1].

\vspace{8cm}
\addpoints
\question[5 Bonus] Evaluate the limit $\lim\limits_{x\to 0}\frac{sin2x}{sin3x}=$
\end{questions}
\end{document}