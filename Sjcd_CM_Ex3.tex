% Exam Template for San Jacinto Community College Math Department courses
%Copyright@Chandi Bhandari
%%%%%%%%%%%%%%%%%%%%%%%%%%%%%%%%%%%%%%%%%%%%%%%%%%%%%%%%%%%%%%%%%%%%%%%%%%%%%%%%%%%%%%%%%%%%%%

% These lines can probably stay unchanged, although you can remove the last
% two packages if you're not making pictures with tikz.
\documentclass[11pt]{exam}
\RequirePackage{amssymb, amsfonts, amsmath, latexsym, verbatim, xspace, setspace}
\RequirePackage{tikz, pgflibraryplotmarks}
\usepackage{graphicx}
\usepackage{venndiagram}
\graphicspath{ {./images/} }

% By default LaTeX uses large margins.  This doesn't work well on exams; problems
% end up in the "middle" of the page, reducing the amount of space for students
% to work on them.
\usepackage[margin=1in]{geometry}


% Here's where you edit the Class, Exam, Date, etc.
\newcommand{\class}{Math 1332 Contemporary Math}
\newcommand{\term}{Fall 2018}
\newcommand{\examnum}{Exam III}
\newcommand{\examdate}{Sept, 2018}
\newcommand{\timelimit}{90 Minutes}

% For an exam, single spacing is most appropriate
\singlespacing
% \onehalfspacing
% \doublespacing

% For an exam, we generally want to turn off paragraph indentation
\parindent 0ex

\begin{document} 

% These commands set up the running header on the top of the exam pages
\pagestyle{head}
\firstpageheader{}{}{}
\runningheader{\class}{\examnum\ - Page \thepage\ of \numpages}{\examdate}
\runningheadrule

\begin{flushright}
\begin{tabular}{p{3.2in} r l}
\textbf{\class} & \textbf{\term} & \textbf{\examnum}\\
\textbf{Class time:} & \textbf{Name (Print):}  \makebox[1.2in]{\hrulefill}\\
%\textbf{Time: \timelimit} % & \makebox[2in]{\hrulefill}

\end{tabular}\\
\end{flushright}
\rule[1ex]{\textwidth}{.1pt}

\hfill


%%%%%%%%%%%%%%%%%%%%%%%%%%%%%%%%%%%%%%%%%%%%%%%%%%%%%%%%%%%%%%%%%%%%%%%%%%%%%%%%%%%%%
%For the further information 
%%%%%%%%%%%%%%%%%%%%%%%%%%%%%%%%%%%%%%%%%%%%%%%%%%%%%%%%%%%%%%%%%%%%%%%%%%%%%%%%%%%%%

\begin{questions}

% Basic question
\addpoints
\question[20] Find the mean, median,and mode for following data and also find outliers if exits: 
\begin{parts}
	\part 7, 3, 3, 11, 12, 3, 4, 14, 6, 4, 3, 53, 4,14, 6
	\vspace{6cm}
	\part 53, 52, 75, 62, 68, 58, 49, 49
\end{parts}
\vspace{9cm}

\newpage
\question[20] Find the five point summary, range and standard deviation of the following.
\begin{parts}
	\part 98, 92, 95, 87, 96, 90, 65
	\vspace{9cm}
	\part 12, 7, 9, 10, 7, 8
	\vspace{9cm}
\end{parts}
\newpage
\addpoints
\question[5] Find the probability distribution table for the sample space when tossing two coins.
\vspace{8cm}
\addpoints
\question[10] Use theoretical method to compute the probability when tossing two coins: 
\begin{parts}
	\part Exactly two head
	\vspace{4cm}
	\part Exactly one Tail
	\vspace{4cm} 
	\part At least one head.
	\vspace{4cm}
	\part No head
\end{parts}

\vspace{9cm}
\addpoints
\question[10]  When forming a committee of three members consisting boys and girls then find 
\begin{parts}
	\part All girs
	\vspace{4cm}
	\part Exactly two boys
	\vspace{4cm} 
	\part At least two boys
	\vspace{4cm}
	\part No girls
	
\end{parts}
\vspace{10cm}

\question[10]  When rolling a die in one hand and tossing a coin in another hand then what is the probability of
\begin{parts}
	\part P(5 and H)
	\vspace{4cm}
	\part P(Even numbers and T)
		
\end{parts}
\vspace{10cm}

\question[10] There are 12 tennis balls in a balls of two different colors 6 red and 4 White balls. John wants to take out two balls from that bag then what the the following probability 
\begin{parts}
	\part P(Red and Red)=P(R and R)
	\vspace{4cm}
	\part P(White and Red)=P(W and R)
	
\end{parts}
\vspace{10cm}

\addpoints
\question[15]  There are 20 cards in a deck of card numbering from 1 through 20 that is $S=\{1, 2, 3, ...,20\}$. And Events A={Even numbers}, B={multiple of 5} and C={3,7, 13} then find the following probability
\begin{parts}
	\part P(A or B)
	\vspace{4cm}
	\part P(A or C)
	\vspace{4cm} 
	\part P(B or C)
	
\end{parts}



\vspace{10cm}
\addpoints
\question[5:] (Bonus) Make a probability distribution table for the number boys in a family of three members. 
\end{questions}
\end{document}