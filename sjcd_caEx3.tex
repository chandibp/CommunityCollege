% Exam Template for San Jacinto Community College Math Department courses
%Copyright@Chandi Bhandari
%%%%%%%%%%%%%%%%%%%%%%%%%%%%%%%%%%%%%%%%%%%%%%%%%%%%%%%%%%%%%%%%%%%%%%%%%%%%%%%%%%%%%%%%%%%%%%

% These lines can probably stay unchanged, although you can remove the last
% two packages if you're not making pictures with tikz.
\documentclass[11pt]{exam}
\RequirePackage{amssymb, amsfonts, amsmath, latexsym, verbatim, xspace, setspace}
\RequirePackage{tikz, pgflibraryplotmarks}
\usepackage{graphicx}

\graphicspath{ {./images/} }

% By default LaTeX uses large margins.  This doesn't work well on exams; problems
% end up in the "middle" of the page, reducing the amount of space for students
% to work on them.
\usepackage[margin=1in]{geometry}


% Here's where you edit the Class, Exam, Date, etc.
\newcommand{\class}{Math 1314 College Algebra}
\newcommand{\term}{Fall 2018}
\newcommand{\examnum}{Exam III}
\newcommand{\examdate}{September, 2018}
\newcommand{\timelimit}{90 Minutes}

% For an exam, single spacing is most appropriate
\singlespacing
% \onehalfspacing
% \doublespacing

% For an exam, we generally want to turn off paragraph indentation
\parindent 0ex

\begin{document} 

% These commands set up the running header on the top of the exam pages
\pagestyle{head}
\firstpageheader{}{}{}
\runningheader{\class}{\examnum\ - Page \thepage\ of \numpages}{\examdate}
\runningheadrule

\begin{flushright}
\begin{tabular}{p{3.2in} r l}
\textbf{\class} & \textbf{\term} & \textbf{\examnum}\\
\textbf{Class Time:} & \textbf{Name (Print):}  \makebox[1.2in]{\hrulefill}\\
%\textbf{Time: \timelimit} % & \makebox[2in]{\hrulefill}

\end{tabular}\\
\end{flushright}
\rule[1ex]{\textwidth}{.1pt}

\hfill


%%%%%%%%%%%%%%%%%%%%%%%%%%%%%%%%%%%%%%%%%%%%%%%%%%%%%%%%%%%%%%%%%%%%%%%%%%%%%%%%%%%%%
%For the further information 
%%%%%%%%%%%%%%%%%%%%%%%%%%%%%%%%%%%%%%%%%%%%%%%%%%%%%%%%%%%%%%%%%%%%%%%%%%%%%%%%%%%%%

\begin{questions}

% Basic question
\addpoints
\question[10]  
\begin{parts}
\part Use the remainder theorem to find the remainder when $f(x)=x^3-5x^2+3x+1$ is divided by $x-1$. \\

\vspace{5cm}
\part Using the above information from 1 a) and factor theorem check that whether $x-1$ is a factor of  $f(x)=x^3-5x^2+3x+1$ or not? 
\end{parts}
\vspace{3cm}
\addpoints
\question[5] Find all the zeros and their multiplicity of the following polynomial. \[f(x)=(x+1)^2(x-1)^3(x^2-10)\]
 
\vspace{9cm}
\addpoints
\question [10] For the polynomial \[ f(x)= x^3-x^2-10x-8\] 
\begin{parts} 
	\part Find all the potential rational zeros of $f(x)$.
	\vspace{4cm}
\part By using the Descarte's Rule of sign how many positive and how many and how many negative solution $f(x)$ has. 
\vspace{9cm}

\end{parts}

\vspace{9cm}
\addpoints
\question [7] Find the end behavior of the following function 
\begin{parts} 
	\part \[f(x)=10x^6-x^5+2x-2\]
	\vspace{4cm}
	\part \[f(x)=-4x^7-x^5+x^3-1\]. 
\end{parts}

\vspace{4cm}
\addpoints
\question [8] Find the zeros and their multiplicity of the following polynomial and sketch the graph \[f(x)=x^2(x-5)(x+3)(x-1)\] 


\vspace{9cm}
\addpoints
\question[15]  For the following function  
$f(x)=\frac{3x}{x^2-x-2}$ 
(Hint: factorize the denominator)
\begin{parts}
\part Domain
\vspace{1cm}
\part x-intercept
\vspace{2cm}
\part y-intercept
\vspace{2cm}
\part Vertical asymptotes
\vspace{3cm}
\part Horizontal or Oblique asymptotes
\vspace{2cm}
\part Sketch the graph
\end{parts}

\vspace{9cm}
\addpoints
\question[15] Find the following
\begin{parts}
	
	\part [8] Check whether \[f(x)=\sqrt{x+6}\] is one to one or not if yes find the its inverse $\bf f^{-1}$ and Also give the $\bf domain$ of $f^{-1}$. 
	\newpage
	\part [7] For the following one-one function find the $f^{-1}$ and Also give the $\bf domain$ of $f^{-1}$.\[f(x)=\frac{x+2}{x-2}\]
\end{parts}
\vspace{12cm}
\addpoints
\question[15] Solve the following

\begin{parts}
\part \[4^{x-2}=2^{3x+3}\]
\vspace{2cm}
\part \[\log_2(x^3+65)=0\]
\vspace{3cm}
\part \[\ln x+\ln x^2=3\]
\vspace{2cm}
\end{parts}

\vspace{8cm}
\addpoints
\question[15] Sketch the following graph by using the transformation techniques
\begin{parts}
	\part \[f(x)=2^{x-1}+2\]
	\vspace{8cm}
	\part \[f(x)=(\frac{1}{3})^{x-2}+3\]
	\vspace{8cm}
	\part  \[f(x)=\log_3(x-1)+2\]
\end{parts}
\vspace{8cm}


\newpage
Bonus
\question[5] Find the future value and interest earned if $\$8000$ is deposited for 9 years at $3\%$ interest compounded quarterly. (Hint: use the formula $A=P(1+\frac{R}{n})^{nT} and I=A-P$).
\vspace{10cm}
\question[5] solve \[\log_2(2x-3)+\log_2(x+1)=1\] 
\end{questions}
\end{document}