% Exam Template for San Jacinto Community College Math Department courses
%Copyright@Chandi Bhandari
%%%%%%%%%%%%%%%%%%%%%%%%%%%%%%%%%%%%%%%%%%%%%%%%%%%%%%%%%%%%%%%%%%%%%%%%%%%%%%%%%%%%%%%%%%%%%%

% These lines can probably stay unchanged, although you can remove the last
% two packages if you're not making pictures with tikz.
\documentclass[11pt]{exam}
\RequirePackage{amssymb, amsfonts, amsmath, latexsym, verbatim, xspace, setspace}
\RequirePackage{tikz, pgflibraryplotmarks}
\usepackage{graphicx}

\graphicspath{ {./images/} }

% By default LaTeX uses large margins.  This doesn't work well on exams; problems
% end up in the "middle" of the page, reducing the amount of space for students
% to work on them.
\usepackage[margin=1in]{geometry}


% Here's where you edit the Class, Exam, Date, etc.
\newcommand{\class}{Math 1314 College Algebra}
\newcommand{\term}{Fall 2018}
\newcommand{\examnum}{Final Exam}
\newcommand{\examdate}{September, 2018}
\newcommand{\timelimit}{90 Minutes}

% For an exam, single spacing is most appropriate
\singlespacing
% \onehalfspacing
% \doublespacing

% For an exam, we generally want to turn off paragraph indentation
\parindent 0ex

\begin{document} 

% These commands set up the running header on the top of the exam pages
\pagestyle{head}
\firstpageheader{}{}{}
\runningheader{\class}{\examnum\ - Page \thepage\ of \numpages}{\examdate}
\runningheadrule

\begin{flushright}
\begin{tabular}{p{3.2in} r l}
\textbf{\class} & \textbf{\term} & \textbf{\examnum}\\
\textbf{Class Time:} & \textbf{Name (Print):}  \makebox[1.2in]{\hrulefill}\\
%\textbf{Time: \timelimit} % & \makebox[2in]{\hrulefill}

\end{tabular}\\
\end{flushright}
\rule[1ex]{\textwidth}{.1pt}

\hfill


%%%%%%%%%%%%%%%%%%%%%%%%%%%%%%%%%%%%%%%%%%%%%%%%%%%%%%%%%%%%%%%%%%%%%%%%%%%%%%%%%%%%%
%For the further information 
%%%%%%%%%%%%%%%%%%%%%%%%%%%%%%%%%%%%%%%%%%%%%%%%%%%%%%%%%%%%%%%%%%%%%%%%%%%%%%%%%%%%%

\begin{questions}
\addpoints
\question Find the product $(5+3i)(2-i)$. 

\vspace{5cm}

\addpoints
\question Solve the following equation by the indicated techniques

\begin{parts}
	\part  $3x^2+5x+2=0$
	\vspace{6cm}
	\part  $x^2-6x+10=0$ 
		\vspace{8cm}
		\part Find the Through (-1,2) and perpendicular to $y=2x-3$.
	\vspace{8cm}
	
\end{parts}



\newpage
\addpoints
\question Word problems


\noaddpoints % If you remove this line, the grading table will show 20 points for this problem.
\begin{parts}
\part A coffee manufacturer wants to market a new blend of coffee that sells for $\$3.90$ per pound by mixing two coffee that sells for $\$2.75$ and $\$5$ per pound, respectively. What amounts of each coffee should be blended to obtain the desire mixture of 100 pound?
\vspace{10cm}
\part Find the future value and interest earned if $\$8000$ is deposited for 9 years at $3\%$ interest compounded quarterly. (Hint: use the formula $A=P(1+\frac{R}{n})^{nT} and I=A-P$)
\end{parts}

\newpage
\addpoints
\question Solve


\noaddpoints % If you remove this line, the grading table will show 20 points for this problem.
\begin{parts}
	\part \[\frac{4}{x-2}-\frac{-3}{x
		+5}=\frac{7}{(x+5)(x-2)}\]
\vspace{8cm}
	\part  \[8-4(2-x)\leq -2x\]

	\vspace{8cm}
	\part \[\left | 3t-2\right |\leq 4\]
	\vspace{8cm}

\end{parts}


\vspace{8cm}
%\newpage

\addpoints
\question  
\begin{parts}
	
	\part  Find the value of $f(-5),f(0),f(3)$ and graph the following piecewise function:
	\begin{center}
		\[f(x)= \begin{cases} 
		\frac{1}{x} & for \quad  x< 0 \\
		\sqrt[3]{x} & for \quad x\geq 0 
		\end{cases}
		\]
	\end{center}
	
	\vspace{8cm}
	Graph the following function. Also give the domain and range
		\part $f(x)=-2(x+1)^4+1$
		\vspace{8cm}
		\part $f(x)=-(x-5)^5+4$
		\vspace{8cm}
		\part  \[f(x)=\frac{1}{x-4}+5\]
		\vspace{8cm}
		\part  \[f(x)=3\sqrt[3]{x-4}+3\]
		\vspace{8cm}
		\part \[f(x)=2^{x-1}+2\]
		\vspace{8cm}
		\part  \[f(x)=\log_3(x-1)+2\]
		
\end{parts}
\vspace{9cm}
\addpoints
\question For the following function 
\[f(x)=(x-5)^3(x+4)^2\] find

\begin{parts}
	\part Find the real zeros and their multiplicity
	\vspace{2cm}
	\part Determine where graph cross or touches (bounces back) at x-axis at the x-intercepts. 
	\vspace{3cm}
	\part Determine where graph cross or touches (bounces back) at x-axis at the x-intercepts.
	\vspace{2cm}
	\part Find the number of maximum turning points
	\vspace{2cm}
	\part Determine  the end behavior
	\vspace{2cm}
	\part Sketch the graph
\end{parts}

\vspace{9cm}
\addpoints
\question 

Solve the system by using inverse matrix method.
\begin{align}
\nonumber
x+3y=&5 \\ 
\nonumber
2x-3y=&-8   
\end{align}
Note that 
\begin{align}
\nonumber
\begin{bmatrix}
1 &   3&  \\
2 &   -3&  
\end{bmatrix}^{-1}=
\begin{bmatrix}
\frac{1}{3} &   \frac{1}{3}&  \\
\frac{2}{9} &   -\frac{1}{9}&  
\end{bmatrix}
\end{align}

\newpage
\addpoints
\question  
\begin{parts}
	\part Use the remainder theorem to find the remainder when $f(x)=x^3-5x^2+3x+1$ is divided by $x-1$. \\
	
	\vspace{5cm}
	\part Using the above information from 1 a) and factor theorem check that whether $x-1$ is a factor of  $f(x)=x^3-5x^2+3x+1$ or not? 
\end{parts}
\vspace{4cm}
\addpoints
\question For the polynomial \[ f(x)= x^3-x^2-10x-8\] 
\begin{parts} 
	\part Find all the potential rational zeros of $f(x)$.
	\vspace{4cm}
	\part By using the Descarte's Rule of sign how many positive and how many and how many negative solution $f(x)$ has. 
	\vspace{9cm}
	
\end{parts}

\vspace{9cm}
\addpoints
\question  
\begin{parts} 
	\part Find the end behavior of the following function \[f(x)=10x^6-x^5+2x-2\]
	\vspace{4cm}
	\part For the following one-one function find the $f^{-1}$ and Also give the $\bf domain$ of $f^{-1}$.\[f(x)=\frac{x+2}{x-2}\] 
\end{parts}

\vspace{9cm}
\addpoints
\question Find the zeros and their multiplicity of the following polynomial and sketch the graph \[f(x)=x^2(x-5)(x+3)(x-1)\] 


\vspace{9cm}
\addpoints
\question  For the following function  
$f(x)=\frac{3x}{x^2-x-2}$ 
(Hint: factorize the denominator as (x-2)(x+1))
\begin{parts}
	\part Domain
	\vspace{1cm}
	\part x-intercept
	\vspace{2cm}
	\part y-intercept
	\vspace{2cm}
	\part Vertical asymptotes
	\vspace{3cm}
	\part Horizontal or Oblique asymptotes
	\vspace{2cm}
	\part Sketch the graph
\end{parts}

\vspace{9cm}

\addpoints
\question Solve the following

\begin{parts}
	\part \[4^{x-2}=2^{3x+3}\]
	\vspace{2cm}
	\part \[\log_2(x^3+65)=0\]
	\vspace{3cm}
	\part \[\ln x+\ln x^2=3\]
	\vspace{2cm}
\end{parts}

\vspace{8cm}

\question solve \[\log_2(2x-3)+\log_2(x+1)=1\] 

\vspace{6cm}
\question Solve the system of equation by matrix method
\begin{align}
\nonumber
x+y=&8 \\ 
\nonumber
x-y=&4   
\end{align}
\end{questions}
\end{document}