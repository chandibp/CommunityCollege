% Exam Template for San Jacinto Community College Math Department courses
%Copyright@Chandi Bhandari
%%%%%%%%%%%%%%%%%%%%%%%%%%%%%%%%%%%%%%%%%%%%%%%%%%%%%%%%%%%%%%%%%%%%%%%%%%%%%%%%%%%%%%%%%%%%%%

% These lines can probably stay unchanged, although you can remove the last
% two packages if you're not making pictures with tikz.
\documentclass[11pt]{exam}
\RequirePackage{amssymb, amsfonts, amsmath, latexsym, verbatim, xspace, setspace}
\RequirePackage{tikz, pgflibraryplotmarks}
\usepackage{graphicx}
\usepackage{venndiagram}
\graphicspath{ {./images/} }

% By default LaTeX uses large margins.  This doesn't work well on exams; problems
% end up in the "middle" of the page, reducing the amount of space for students
% to work on them.
\usepackage[margin=1in]{geometry}


% Here's where you edit the Class, Exam, Date, etc.
\newcommand{\class}{Math 1332 Contemporary Math}
\newcommand{\term}{Fall 2018}
\newcommand{\examnum}{Exam II}
\newcommand{\examdate}{Sept, 2018}
\newcommand{\timelimit}{90 Minutes}

% For an exam, single spacing is most appropriate
\singlespacing
% \onehalfspacing
% \doublespacing

% For an exam, we generally want to turn off paragraph indentation
\parindent 0ex

\begin{document} 

% These commands set up the running header on the top of the exam pages
\pagestyle{head}
\firstpageheader{}{}{}
\runningheader{\class}{\examnum\ - Page \thepage\ of \numpages}{\examdate}
\runningheadrule

\begin{flushright}
\begin{tabular}{p{3.2in} r l}
\textbf{\class} & \textbf{\term} & \textbf{\examnum}\\
\textbf{G-Number-} & \textbf{Name (Print):}  \makebox[1.2in]{\hrulefill}\\
%\textbf{Time: \timelimit} % & \makebox[2in]{\hrulefill}

\end{tabular}\\
\end{flushright}
\rule[1ex]{\textwidth}{.1pt}

\hfill


%%%%%%%%%%%%%%%%%%%%%%%%%%%%%%%%%%%%%%%%%%%%%%%%%%%%%%%%%%%%%%%%%%%%%%%%%%%%%%%%%%%%%
%For the further information 
%%%%%%%%%%%%%%%%%%%%%%%%%%%%%%%%%%%%%%%%%%%%%%%%%%%%%%%%%%%%%%%%%%%%%%%%%%%%%%%%%%%%%

\begin{questions}

% Basic question
\addpoints
\question[4] Define the following with example: 
\begin{parts}
	\part Correlation
	\vspace{4cm}
	\part Scatter plot
\end{parts}
\vspace{5cm}

\question[5] Draw the scatter plot showing 
\begin{parts}
	\part Positive Correlation
	\vspace{5cm}
	\part Negative Correlation
	\vspace{4cm}
\end{parts}
%\newpage
\addpoints
\question[8] Define the following with example: 
\begin{parts}
	\part Population and Sample
	\vspace{4cm}
	\part Sampling
	\vspace{4cm} 
	\part Simple random sampling, Systematic sampling and Stratified random sampling.
	\vspace{4cm}
	\part Treatment and Control group
\end{parts}
\newpage
\addpoints
\question[6] Make a frequency table and then find cumulative  and relative frequency for the following data\\
A, C, D, F, H, C, C, C, A, C, H, D, D, C, A, F, H, H, H, H, A, A, D, D, A, A, C, D. 
\vspace{10cm}
\addpoints
\question[15] Make a frequency table and relative frequency also present them in  Bar Diagram and Pie-chart for the following data\\
13, 4, 7, 11, 4, 13, 13, 3, 7, 3, 13, 13, 7, 7, 7, 8, 4, 13, 3, 8, 11, 4, 13, 7, 7, 4, 13, 11, 11, 7, 8, 4, 13, 4, 3, 7, 3, 4, 8, 13, 11, 8,13, 11, 4, 4, 4, 13, 3, 8, 4, 7, 13,11, 3, 7, 3, 13. 
\vspace{10cm}
\question[6] If interest rates stay at $4\%$ APR and I continue to make my monthly $\$25$ deposit into my retirement plan, how much total money I would have after 30 years. 
\vspace{10cm}

\question[8] Compute the total and Annual returns for: you paid $\$8000$ for a municipal bond. When it matures after 20 years, you received $\$12,500$.  
\vspace{10cm}
\question[6] Your goal is to create a college fund for your child. Suppose you find a fund that offers an APR of $5\%$ How much should you deposit monthly to accumulate $\$170,000$ in 15 years? 
\vspace{10cm}
\question[10] Compute the total amount paid and what percent is principal and what percentage is interest paid when you borrowed $\$100,000$ for a period of 30 years at a fixed APR $5.5\%$.  
\vspace{10cm}
\question[15] Compare the monthly payment, Total payment for the two different option a fixed loan amount. Compare the pros and Cons of each loan option.\\
You need $\$400,000$ loan. \\
Option 1: a 30-year loan at an APR of $8\%$\\
Option 2: a 15-year loan at an APR of $7.5\%$ 
\newpage
\question [10] Convert the following data into continuous data and express them in the histogram \begin{table}[h]                           
	\centering
	\begin{tabular}{|c|c|}
		\hline
		\multicolumn{2}{c|}{\textbf{Age of Academy Award-wining Male actor at Time of award}} \\ \hline
		
		Age    &   Number of actors         \\ \hline
		20-29 &   1             \\ \hline          
		30-39    &   11  \\ \hline
		40-49          &  14          \\ \hline
		50-59  &   13              \\ \hline
		60-69       &  6     \\ \hline
		70-79          &   2         \\ \hline
		
	\end{tabular}
	\caption{Oscar-winning Male actors}
	\label{tab:msg1}                            
	
\end{table}\\
\newpage

\question[7] The following data represents the number of boys and girls attended at San Jacinto College from 2010 to 2015 in College Comtemporary math classes. Make a multiple bar diagram (single or double) for these data, with vertical axes representing the number of students running from 1 to 15.  \begin{table}[h]                           
	\centering
	\begin{tabular}{|c|c|c|}
		\hline
		\multicolumn{3}{c|}{\textbf{Number of boys and girls from 2010-2015}} \\ \hline
		
		Academic year    &   Number of boys   &   Number of girls      \\ \hline
		2010 &   12 &   14             \\ \hline          
		2011    &   9 &   16  \\ \hline
		2012       &  14 &   10           \\ \hline
		2013  &   13 &   8              \\ \hline
		2014       &  6  &   10   \\ \hline
		2015        &   6 &   12          \\ \hline
		
	\end{tabular}
	\caption{Students log table in Contemporary math class}
	\label{tab:msg1}                            
	
\end{table}
\vspace{10cm}
\addpoints
\question[6 Bonus:] Suppose that on the January 1, 2018 you had a balance of $\$10,000$ on Chase Credit Card which charges APR $20\%$, you want to paid the balance off in 5 years. Then how much is your monthly payment. 
\end{questions}
\end{document}