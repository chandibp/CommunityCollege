\documentclass{exam}
\usepackage[utf8]{inputenc}
 
\begin{document}
 
\begin{center}
\fbox{\fbox{\parbox{5.5in}{\centering
Math 1314  \hfill Exam 3 \\Answer the questions in the spaces provided.}}}
\end{center}
 
\vspace{5mm}
 
\makebox[\textwidth]{Name and Time:\enspace\hrulefill}
 
\vspace{5mm}
 
\makebox[\textwidth]{G-Number:\enspace\hrulefill}
 
\begin{questions}
	
	\question 
	\subitem a) Use the remainder theorem to find the remainder when $f(x)=3x^4-6x^3-5x+10$ is divided by $x-2$. \\
	 
	
	\subitem b)Using the above information from 1 a) and factor theorem check that whether $x-2$ is a factor of $f(x)=3x^4-6x^3-5x+10$ or not?
		\hfill\enspace\hrulefill
		%\vspace{\stretch{1}}
	
	\clearpage
	\question List all the potential rational zero of $f(x)=2x^5-x^3-2x^2+12$
	\hfill\enspace\hrulefill
	\vspace{\stretch{1}}
	\question Use Descarte's rule of sing to determine how many positive solution $f(x)=2x^5-x^3-2x^2+12$ has.
	\hfill\enspace\hrulefill
	\vspace{\stretch{1}}
	\clearpage
	
	\question Find the composite a) $f\circ f(x)$ b) $f\circ g(x)$ c) $f\circ g (0)$ for the function $f(x)=3x+1 \quad\mbox{and} \quad g(x)=x^2$.
	\hfill\enspace\hrulefill 
	\vspace{\stretch{1}}
	
	\clearpage
	\question Is $f(x) =\frac{2}{3+x}$ is one to one? Give reason. 
	\hfill\enspace\hrulefill 
	\vspace{\stretch{1}}
	
	
	
	
	\question Find the inverse of the following one-one function, $f(x)=\frac{4}{2-x} $
		\hfill\enspace\hrulefill 
		\vspace{\stretch{1}}
		
\clearpage
		
	\question Solve for the x,
	\subitem a) \(2^{-x}=16\) \hfill\enspace\hrulefill 
	\vspace{\stretch{0.3}}
	
	\subitem b) \((\frac{1}{5})^{x}=\frac{1}{25}\) \hfill\enspace\hrulefill 
	\vspace{\stretch{0.3}}
	
	\subitem c) \(9^{2x}27^{x^2}=3^{-1}\) 
	\hfill\enspace\hrulefill 
	\vspace{\stretch{0.3}}
	
	\clearpage
	\question find the exact value of 
	\subitem a) \(log_5(25)\) \hfill\enspace\hrulefill 
	\vspace{\stretch{0.1}}
	
	\subitem b) \(log_{10}(\sqrt{10})\) \hfill\enspace\hrulefill 
	\vspace{\stretch{0.1}}
	
	\subitem c) \(log_{\frac{1}{3}}(9)\) \hfill\enspace\hrulefill 
	\vspace{\stretch{0.1}}
	
	
	\clearpage
	\question Use the properties of logarithm to express as sum and difference 
	\subitem a) \(log_5(25x)\) \hfill\enspace\hrulefill 
	\vspace{\stretch{0.1}}
	
	\subitem b) \(log_{2}(z^3)\) \hfill\enspace\hrulefill 
	\vspace{\stretch{0.1}}
	
	\subitem c) \(ln(xe^x)\) \hfill\enspace\hrulefill 
	\vspace{\stretch{0.1}}
\end{questions}
\clearpage
\end{document}