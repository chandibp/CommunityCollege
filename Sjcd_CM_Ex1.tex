% Exam Template for San Jacinto Community College Math Department courses
%Copyright@Chandi Bhandari
%%%%%%%%%%%%%%%%%%%%%%%%%%%%%%%%%%%%%%%%%%%%%%%%%%%%%%%%%%%%%%%%%%%%%%%%%%%%%%%%%%%%%%%%%%%%%%

% These lines can probably stay unchanged, although you can remove the last
% two packages if you're not making pictures with tikz.
\documentclass[11pt]{exam}
\RequirePackage{amssymb, amsfonts, amsmath, latexsym, verbatim, xspace, setspace}
\RequirePackage{tikz, pgflibraryplotmarks}
\usepackage{graphicx}
\usepackage{venndiagram}
\graphicspath{ {./images/} }

% By default LaTeX uses large margins.  This doesn't work well on exams; problems
% end up in the "middle" of the page, reducing the amount of space for students
% to work on them.
\usepackage[margin=1in]{geometry}


% Here's where you edit the Class, Exam, Date, etc.
\newcommand{\class}{Math 1332 Contemporary Math}
\newcommand{\term}{Fall 2018}
\newcommand{\examnum}{Exam I}
\newcommand{\examdate}{Sept, 2018}
\newcommand{\timelimit}{90 Minutes}

% For an exam, single spacing is most appropriate
\singlespacing
% \onehalfspacing
% \doublespacing

% For an exam, we generally want to turn off paragraph indentation
\parindent 0ex

\begin{document} 

% These commands set up the running header on the top of the exam pages
\pagestyle{head}
\firstpageheader{}{}{}
\runningheader{\class}{\examnum\ - Page \thepage\ of \numpages}{\examdate}
\runningheadrule

\begin{flushright}
\begin{tabular}{p{3.2in} r l}
\textbf{\class} & \textbf{\term} & \textbf{\examnum}\\
\textbf{G-Number-} & \textbf{Name (Print):}  \makebox[1.2in]{\hrulefill}\\
%\textbf{Time: \timelimit} % & \makebox[2in]{\hrulefill}

\end{tabular}\\
\end{flushright}
\rule[1ex]{\textwidth}{.1pt}

\hfill


%%%%%%%%%%%%%%%%%%%%%%%%%%%%%%%%%%%%%%%%%%%%%%%%%%%%%%%%%%%%%%%%%%%%%%%%%%%%%%%%%%%%%
%For the further information 
%%%%%%%%%%%%%%%%%%%%%%%%%%%%%%%%%%%%%%%%%%%%%%%%%%%%%%%%%%%%%%%%%%%%%%%%%%%%%%%%%%%%%

\begin{questions}

% Basic question
\addpoints
\question[12] Identify the premises, conclusion, and the type of Fallacies 

\begin{parts}
	\part I ate oysters for dinner and later that night I had a nightmare. Oyster caused my nightmare
\vspace{5cm}
	\part My uncle never drank alcohol and lived to be 93 years old. That's how I know that avoiding alcohol leads to greater longevity.
	\vspace{5cm}
	\part Apple's iPhones outsell all other smart phones, so they must be the best smart phones on the market.  
\end{parts}

\vspace{5cm}
%\newpage
\addpoints
\question[20] For the two statements p, q find the following logical argument truth table for Negation:  not p ($\sim{p}$), Conjunction: p and q ($p\land q$), Disjunction: p or q ($p\lor q$), Conditional: if p then q ($p\implies q$), Inverse: if not p then not q ($\sim{p}\implies \sim q$), Converse: if q then p ($q\implies p$), Contrapositive: if not q then not p ($\sim{q}\implies \sim p$) .



 
\newpage
%\newpage
\addpoints
\question[15] A movie critics reviewed 36 films: 12 were documentaries and 24 were feature films. She gave favorable reviews to 18 of the 24 were feature films. She gave favorable reviews to 8 of the documentaries and unfavorable reviews to 6 of the feature films.
\begin{parts}
\part  Make a two-way table summarizing the reviews.
\vspace{5cm}
\part  Make a Venn diagram from the table in part (a).
\vspace{5cm}
\part How many documentaries received unfavorable reviews? 
\vspace{5cm}
\part How many feature films received favorable reviews?
\end{parts}

\vspace{15cm}
\addpoints
\question [10] Following is the Venn-diagram of the people at conference. Use the Venn Diagram to answer the following questions

\begin{venndiagram3sets}[labelOnlyAB={20},
	labelOnlyAC={8},
	labelOnlyBC={16},
	labelABC={11},
	labelOnlyA={4},
	labelOnlyC={6},
	labelOnlyB={9},
	labelNotABC={3},
	radius=3cm,
	overlap=2.5cm]
	\setpostvennhook
	{
		\draw[<-] (labelA) -- ++(135:3cm) node[above] {Woman};
		\draw[<-] (labelB) -- ++(45:3cm) node[above] {College degree};
		\draw[<-] (labelC) -- ++(-90:3cm) node[below] {Currently employed};
	}
\end{venndiagram3sets}




\begin{parts}
\part  How many people at the conference are employed men with a college degree?
\vspace{1cm}
\part  How many people at the conference are employed men?
\vspace{1cm}
\part   How many people at the conference are unemployed woman?
\vspace{1cm}
\part   How many women are at the conference?
\vspace{1cm}

\end{parts}



\newpage
\addpoints
\question[10] Compute the compounded interest for the following problems.


\noaddpoints % If you remove this line, the grading table will show 20 points for this problem.
\begin{parts}
	\part[5] $\$5000$ is invested for 10 years with an APR of $2\%$ and quarterly compounding.  

\vspace{10cm}
\part[5] $\$4000$ is invested for 20 years with an APR of $2\%$ and monthly compounding.  
\end{parts}
\vspace{10cm}
\question[10] For the following income and expenses compute the Net Cash flow: \\
Income:\\
Part-time Job = $\$650$ per month\\
College fund from grandparents: $\$400$ per month\\
Scholarship: $\$6000$ per year. \\

Expenses: \\
Rent: $\$500$ per month.\\
Groceries: $\$60 $ per week.\\ 
Tuition and Fees: $\$3600$ twice a year.\\
Incidentals: $\$120$ per week \\

\vspace{10cm}
\question[15] You have a choice of two health insurance plans: \\ 
Plan A:\\
Monthly premium:   $\$300$ \\
Annual deductible:   $\$5000$ \\
Office visit  co-payment: $\$25$. \\
Emergency room co-payment: $\$500$ \\
Surgical operations co-payments: $\$250$ per year. \\

Plan B:\\
A annual premium:   $\$700$ \\
Annual deductible:   $\$1500$ \\
Office visit  co-payment: $\$25$\\ 
Emergency room co-payment: $\$200$ \\
Surgical operations co-payments: no co-payment.\\

Suppose that during a one-year period your family received the following medical bills. \\
Service and Costs are as following: \\
Jan. 23: Emergency room $\$800$\\
Feb. 14: Office Visit $\$100$\\
Apr. 13: Surgery      $\$1400$\\
June 14: Surgery      $\$7500$\\
July 1: Office visit   $\$100$\\
September 23 Emergency room $\$1200$\\
\begin{parts}
	\part[5] Determine your annual health care expenses if you have plan A.  
	
	\vspace{8cm}
	\part[5] Determine your annual health care expenses if you have plan B.
	\vspace{8cm}
	\part[5]Determine your health care expenses if you have no health insurance. Does have no health insurance create any other risks for you? Explain.
\end{parts}

\addpoints
\question[8] Determine the validity of the following statement. \\
Premise: If taxes are cut the U.S. Government will have less revenue. \\
Premises: If there is less revenue, then the federal deficit will be larger.\\
Conclusion: Tax cuts will lead to a larger federal deficit.\\
\vspace{10cm}

Bonus:
\question[5] Determine the truth of the following premises, and check whether it is sound or not, Also present in the Venn-diagram. \\
Premise: All queens are woman. \\
Premises: Meryl Streep is a woman.\\
Conclusion: Meryl Streep is a queen. \\
\end{questions}
\end{document}